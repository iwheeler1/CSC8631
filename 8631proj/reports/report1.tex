% Options for packages loaded elsewhere
\PassOptionsToPackage{unicode}{hyperref}
\PassOptionsToPackage{hyphens}{url}
%
\documentclass[
]{article}
\usepackage{lmodern}
\usepackage{amssymb,amsmath}
\usepackage{ifxetex,ifluatex}
\ifnum 0\ifxetex 1\fi\ifluatex 1\fi=0 % if pdftex
  \usepackage[T1]{fontenc}
  \usepackage[utf8]{inputenc}
  \usepackage{textcomp} % provide euro and other symbols
\else % if luatex or xetex
  \usepackage{unicode-math}
  \defaultfontfeatures{Scale=MatchLowercase}
  \defaultfontfeatures[\rmfamily]{Ligatures=TeX,Scale=1}
\fi
% Use upquote if available, for straight quotes in verbatim environments
\IfFileExists{upquote.sty}{\usepackage{upquote}}{}
\IfFileExists{microtype.sty}{% use microtype if available
  \usepackage[]{microtype}
  \UseMicrotypeSet[protrusion]{basicmath} % disable protrusion for tt fonts
}{}
\makeatletter
\@ifundefined{KOMAClassName}{% if non-KOMA class
  \IfFileExists{parskip.sty}{%
    \usepackage{parskip}
  }{% else
    \setlength{\parindent}{0pt}
    \setlength{\parskip}{6pt plus 2pt minus 1pt}}
}{% if KOMA class
  \KOMAoptions{parskip=half}}
\makeatother
\usepackage{xcolor}
\IfFileExists{xurl.sty}{\usepackage{xurl}}{} % add URL line breaks if available
\IfFileExists{bookmark.sty}{\usepackage{bookmark}}{\usepackage{hyperref}}
\hypersetup{
  pdftitle={CSC8631 report},
  pdfauthor={Isaac Wheeler},
  hidelinks,
  pdfcreator={LaTeX via pandoc}}
\urlstyle{same} % disable monospaced font for URLs
\usepackage[margin=1in]{geometry}
\usepackage{color}
\usepackage{fancyvrb}
\newcommand{\VerbBar}{|}
\newcommand{\VERB}{\Verb[commandchars=\\\{\}]}
\DefineVerbatimEnvironment{Highlighting}{Verbatim}{commandchars=\\\{\}}
% Add ',fontsize=\small' for more characters per line
\usepackage{framed}
\definecolor{shadecolor}{RGB}{248,248,248}
\newenvironment{Shaded}{\begin{snugshade}}{\end{snugshade}}
\newcommand{\AlertTok}[1]{\textcolor[rgb]{0.94,0.16,0.16}{#1}}
\newcommand{\AnnotationTok}[1]{\textcolor[rgb]{0.56,0.35,0.01}{\textbf{\textit{#1}}}}
\newcommand{\AttributeTok}[1]{\textcolor[rgb]{0.77,0.63,0.00}{#1}}
\newcommand{\BaseNTok}[1]{\textcolor[rgb]{0.00,0.00,0.81}{#1}}
\newcommand{\BuiltInTok}[1]{#1}
\newcommand{\CharTok}[1]{\textcolor[rgb]{0.31,0.60,0.02}{#1}}
\newcommand{\CommentTok}[1]{\textcolor[rgb]{0.56,0.35,0.01}{\textit{#1}}}
\newcommand{\CommentVarTok}[1]{\textcolor[rgb]{0.56,0.35,0.01}{\textbf{\textit{#1}}}}
\newcommand{\ConstantTok}[1]{\textcolor[rgb]{0.00,0.00,0.00}{#1}}
\newcommand{\ControlFlowTok}[1]{\textcolor[rgb]{0.13,0.29,0.53}{\textbf{#1}}}
\newcommand{\DataTypeTok}[1]{\textcolor[rgb]{0.13,0.29,0.53}{#1}}
\newcommand{\DecValTok}[1]{\textcolor[rgb]{0.00,0.00,0.81}{#1}}
\newcommand{\DocumentationTok}[1]{\textcolor[rgb]{0.56,0.35,0.01}{\textbf{\textit{#1}}}}
\newcommand{\ErrorTok}[1]{\textcolor[rgb]{0.64,0.00,0.00}{\textbf{#1}}}
\newcommand{\ExtensionTok}[1]{#1}
\newcommand{\FloatTok}[1]{\textcolor[rgb]{0.00,0.00,0.81}{#1}}
\newcommand{\FunctionTok}[1]{\textcolor[rgb]{0.00,0.00,0.00}{#1}}
\newcommand{\ImportTok}[1]{#1}
\newcommand{\InformationTok}[1]{\textcolor[rgb]{0.56,0.35,0.01}{\textbf{\textit{#1}}}}
\newcommand{\KeywordTok}[1]{\textcolor[rgb]{0.13,0.29,0.53}{\textbf{#1}}}
\newcommand{\NormalTok}[1]{#1}
\newcommand{\OperatorTok}[1]{\textcolor[rgb]{0.81,0.36,0.00}{\textbf{#1}}}
\newcommand{\OtherTok}[1]{\textcolor[rgb]{0.56,0.35,0.01}{#1}}
\newcommand{\PreprocessorTok}[1]{\textcolor[rgb]{0.56,0.35,0.01}{\textit{#1}}}
\newcommand{\RegionMarkerTok}[1]{#1}
\newcommand{\SpecialCharTok}[1]{\textcolor[rgb]{0.00,0.00,0.00}{#1}}
\newcommand{\SpecialStringTok}[1]{\textcolor[rgb]{0.31,0.60,0.02}{#1}}
\newcommand{\StringTok}[1]{\textcolor[rgb]{0.31,0.60,0.02}{#1}}
\newcommand{\VariableTok}[1]{\textcolor[rgb]{0.00,0.00,0.00}{#1}}
\newcommand{\VerbatimStringTok}[1]{\textcolor[rgb]{0.31,0.60,0.02}{#1}}
\newcommand{\WarningTok}[1]{\textcolor[rgb]{0.56,0.35,0.01}{\textbf{\textit{#1}}}}
\usepackage{graphicx,grffile}
\makeatletter
\def\maxwidth{\ifdim\Gin@nat@width>\linewidth\linewidth\else\Gin@nat@width\fi}
\def\maxheight{\ifdim\Gin@nat@height>\textheight\textheight\else\Gin@nat@height\fi}
\makeatother
% Scale images if necessary, so that they will not overflow the page
% margins by default, and it is still possible to overwrite the defaults
% using explicit options in \includegraphics[width, height, ...]{}
\setkeys{Gin}{width=\maxwidth,height=\maxheight,keepaspectratio}
% Set default figure placement to htbp
\makeatletter
\def\fps@figure{htbp}
\makeatother
\setlength{\emergencystretch}{3em} % prevent overfull lines
\providecommand{\tightlist}{%
  \setlength{\itemsep}{0pt}\setlength{\parskip}{0pt}}
\setcounter{secnumdepth}{-\maxdimen} % remove section numbering

\title{CSC8631 report}
\author{Isaac Wheeler}
\date{26/11/2020}

\begin{document}
\maketitle

In this project, I am going to be providing some analysis of the video
stats data; asking a few different questions and trying to find any sort
of pattern within these. Video stats are only provided for years 3-7 so
this is where the my analysis will be based upon. As mentioned in the
executive summary, the data in these files is already in a very usable
state. As such, data preparation takes place before I make different
parts of my analysis and are included in the code chunks and briefly
explained.

This report takes the form of asking a few questions about the data.
I'll indicate what I am looking into, prepare the data to analyse, plot
and analyse, before evaluating in conjunction with my understanding of
what the client is looking for in the data.

\hypertarget{video-engagement-across-years}{%
\section{Video engagement across
years}\label{video-engagement-across-years}}

\hypertarget{watch-percentages}{%
\subsection{Watch percentages}\label{watch-percentages}}

The first question I investigated is if there is any significant
differences between years in terms of engagement with the videos. This
was by done by looking into how the percentage watched (up to different
percentages of the video) changed, if at all, across the years.

In order to do this, I knocked up a quick function that would work out
across all the videos the average percentage of views that made it to
differing thresholds of the video. These thresholds were 5\%, 10\%,
25\%, 50\%, 75\%, 95\% and 100\%, and the data were all provided in the
video stats csv files.

\begin{Shaded}
\begin{Highlighting}[]
\NormalTok{helper.avgper <-}\StringTok{ }\ControlFlowTok{function}\NormalTok{(y)}
\NormalTok{\{}
\NormalTok{  temp =}\StringTok{ }\KeywordTok{rep}\NormalTok{(}\OtherTok{NULL}\NormalTok{,}\DecValTok{7}\NormalTok{)}
  \ControlFlowTok{for}\NormalTok{ (i }\ControlFlowTok{in} \DecValTok{9}\OperatorTok{:}\DecValTok{15}\NormalTok{)}
\NormalTok{  \{}
\NormalTok{    temp[i}\DecValTok{-8}\NormalTok{] =}\StringTok{ }\KeywordTok{sum}\NormalTok{(y[i])}\OperatorTok{/}\DecValTok{13}
\NormalTok{  \}}
  \KeywordTok{return}\NormalTok{(temp)}
\NormalTok{\}}
\end{Highlighting}
\end{Shaded}

This simple helper function will, when provided with a video stats file
of the same format, work out the average percentages watched for the
year, and return it as a vector. This function can be found in the
helpers file of the repository, along with one other that we're going to
use later.

So to carry out the analysis, we simply need to run the function for all
of our years and then plot it.

\begin{Shaded}
\begin{Highlighting}[]
\NormalTok{y3a=}\KeywordTok{helper.avgper}\NormalTok{(y3)}
\NormalTok{y4a=}\KeywordTok{helper.avgper}\NormalTok{(y4)}
\NormalTok{y5a=}\KeywordTok{helper.avgper}\NormalTok{(y5)}
\NormalTok{y6a=}\KeywordTok{helper.avgper}\NormalTok{(y6)}
\NormalTok{y7a=}\KeywordTok{helper.avgper}\NormalTok{(y7)}
\end{Highlighting}
\end{Shaded}

We store each vector on its own. For the benefit of readability alone I
will put this all into a dataframe also.

\begin{Shaded}
\begin{Highlighting}[]
\NormalTok{a5p =}\StringTok{ }\KeywordTok{c}\NormalTok{(y3a[}\DecValTok{1}\NormalTok{],y4a[}\DecValTok{1}\NormalTok{],y5a[}\DecValTok{1}\NormalTok{],y6a[}\DecValTok{1}\NormalTok{],y7a[}\DecValTok{1}\NormalTok{])}
\NormalTok{a10p =}\StringTok{ }\KeywordTok{c}\NormalTok{(y3a[}\DecValTok{2}\NormalTok{],y4a[}\DecValTok{2}\NormalTok{],y5a[}\DecValTok{2}\NormalTok{],y6a[}\DecValTok{2}\NormalTok{],y7a[}\DecValTok{2}\NormalTok{])}
\NormalTok{a25p =}\StringTok{ }\KeywordTok{c}\NormalTok{(y3a[}\DecValTok{3}\NormalTok{],y4a[}\DecValTok{3}\NormalTok{],y5a[}\DecValTok{3}\NormalTok{],y6a[}\DecValTok{3}\NormalTok{],y7a[}\DecValTok{3}\NormalTok{])}
\NormalTok{a50p =}\StringTok{ }\KeywordTok{c}\NormalTok{(y3a[}\DecValTok{4}\NormalTok{],y4a[}\DecValTok{4}\NormalTok{],y5a[}\DecValTok{4}\NormalTok{],y6a[}\DecValTok{4}\NormalTok{],y7a[}\DecValTok{4}\NormalTok{])}
\NormalTok{a75p =}\StringTok{ }\KeywordTok{c}\NormalTok{(y3a[}\DecValTok{5}\NormalTok{],y4a[}\DecValTok{5}\NormalTok{],y5a[}\DecValTok{5}\NormalTok{],y6a[}\DecValTok{5}\NormalTok{],y7a[}\DecValTok{5}\NormalTok{])}
\NormalTok{a95p =}\StringTok{ }\KeywordTok{c}\NormalTok{(y3a[}\DecValTok{6}\NormalTok{],y4a[}\DecValTok{6}\NormalTok{],y5a[}\DecValTok{6}\NormalTok{],y6a[}\DecValTok{6}\NormalTok{],y7a[}\DecValTok{6}\NormalTok{])}
\NormalTok{a100p =}\StringTok{ }\KeywordTok{c}\NormalTok{(y3a[}\DecValTok{7}\NormalTok{],y4a[}\DecValTok{7}\NormalTok{],y5a[}\DecValTok{7}\NormalTok{],y6a[}\DecValTok{7}\NormalTok{],y7a[}\DecValTok{7}\NormalTok{])}

\NormalTok{labe =}\StringTok{ }\KeywordTok{c}\NormalTok{(}\StringTok{"5%"}\NormalTok{,}\StringTok{"10%"}\NormalTok{,}\StringTok{"25%"}\NormalTok{,}\StringTok{"50%"}\NormalTok{,}\StringTok{"75%"}\NormalTok{,}\StringTok{"95%"}\NormalTok{,}\StringTok{"100%"}\NormalTok{)}
\NormalTok{(}\DataTypeTok{viddf =} \KeywordTok{data.frame}\NormalTok{(}\DataTypeTok{year=}\DecValTok{3}\OperatorTok{:}\DecValTok{7}\NormalTok{,}\StringTok{"5%"}\NormalTok{=a5p,}\StringTok{"10%"}\NormalTok{=a10p,}\StringTok{"25%"}\NormalTok{=a25p,}
                    \StringTok{"50%"}\NormalTok{=a50p,}\StringTok{"75%"}\NormalTok{=a75p,}\StringTok{"95%"}\NormalTok{=a95p,}\StringTok{"100%"}\NormalTok{=a100p))}
\end{Highlighting}
\end{Shaded}

\begin{verbatim}
##   year      X5.     X10.     X25.     X50.     X75.     X95.    X100.
## 1    3 74.25846 72.95462 71.10846 68.48462 66.63923 64.25231 56.34308
## 2    4 73.45077 71.77538 69.59462 66.64692 64.47923 62.24154 55.43231
## 3    5 78.95000 77.57231 75.00923 72.51154 70.70231 68.69308 60.56000
## 4    6 78.67692 76.73000 73.97385 71.03385 68.99154 66.77308 57.92846
## 5    7 74.65308 72.84308 69.24000 66.28231 64.25308 62.05000 55.65769
\end{verbatim}

There are some interesting things to draw already from this data. Year
five was a good year for video engagement in terms of overall
percentages, whereas year four was actually the worst. We can also see
that even for the most engaged years, almost 20\% of video views do not
even get 5\% into the video. Further, there is a consistent drop off of
views that overall means videos lose about 15-20\% of the views as the
video goes on to the finish. In order to visualise this fully, we're
going to want to plot our data. Notice how I pointed out that the reason
for the above dataframe was readability alone; we are going to use
something else for our plot. GGplot requires that you use a data frame
when you plot data, so why do we need another data frame when we've got
a perfectly good one sitting around already. Three words; grouped bar
plot.

\begin{Shaded}
\begin{Highlighting}[]
\NormalTok{year =}\StringTok{ }\KeywordTok{c}\NormalTok{(}\KeywordTok{rep}\NormalTok{(}\StringTok{"year 3"}\NormalTok{,}\DecValTok{7}\NormalTok{),}\KeywordTok{rep}\NormalTok{(}\StringTok{"year 4"}\NormalTok{,}\DecValTok{7}\NormalTok{),}\KeywordTok{rep}\NormalTok{(}\StringTok{"year 5"}\NormalTok{,}\DecValTok{7}\NormalTok{),}
         \KeywordTok{rep}\NormalTok{(}\StringTok{"year 6"}\NormalTok{,}\DecValTok{7}\NormalTok{),}\KeywordTok{rep}\NormalTok{(}\StringTok{"year 7"}\NormalTok{,}\DecValTok{7}\NormalTok{))}
\NormalTok{perwatched =}\StringTok{ }\KeywordTok{rep}\NormalTok{(}\KeywordTok{c}\NormalTok{(}\StringTok{"5%"}\NormalTok{,}\StringTok{"10%"}\NormalTok{,}\StringTok{"25%"}\NormalTok{,}\StringTok{"50%"}\NormalTok{,}\StringTok{"75%"}\NormalTok{,}\StringTok{"95%"}\NormalTok{,}\StringTok{"100%"}\NormalTok{),}\DecValTok{5}\NormalTok{)}
\NormalTok{values =}\StringTok{ }\KeywordTok{c}\NormalTok{(y3a,y4a,y5a,y6a,y7a)}
\NormalTok{(}\DataTypeTok{datas =} \KeywordTok{data.frame}\NormalTok{(year,perwatched,values))}
\end{Highlighting}
\end{Shaded}

\begin{verbatim}
##      year perwatched   values
## 1  year 3         5% 74.25846
## 2  year 3        10% 72.95462
## 3  year 3        25% 71.10846
## 4  year 3        50% 68.48462
## 5  year 3        75% 66.63923
## 6  year 3        95% 64.25231
## 7  year 3       100% 56.34308
## 8  year 4         5% 73.45077
## 9  year 4        10% 71.77538
## 10 year 4        25% 69.59462
## 11 year 4        50% 66.64692
## 12 year 4        75% 64.47923
## 13 year 4        95% 62.24154
## 14 year 4       100% 55.43231
## 15 year 5         5% 78.95000
## 16 year 5        10% 77.57231
## 17 year 5        25% 75.00923
## 18 year 5        50% 72.51154
## 19 year 5        75% 70.70231
## 20 year 5        95% 68.69308
## 21 year 5       100% 60.56000
## 22 year 6         5% 78.67692
## 23 year 6        10% 76.73000
## 24 year 6        25% 73.97385
## 25 year 6        50% 71.03385
## 26 year 6        75% 68.99154
## 27 year 6        95% 66.77308
## 28 year 6       100% 57.92846
## 29 year 7         5% 74.65308
## 30 year 7        10% 72.84308
## 31 year 7        25% 69.24000
## 32 year 7        50% 66.28231
## 33 year 7        75% 64.25308
## 34 year 7        95% 62.05000
## 35 year 7       100% 55.65769
\end{verbatim}

\begin{Shaded}
\begin{Highlighting}[]
\NormalTok{datas}\OperatorTok{$}\NormalTok{perwatched =}\StringTok{ }\KeywordTok{factor}\NormalTok{(datas}\OperatorTok{$}\NormalTok{perwatched, }\DataTypeTok{levels=}\KeywordTok{c}\NormalTok{(}\StringTok{"5%"}\NormalTok{,}\StringTok{"10%"}\NormalTok{,}\StringTok{"25%"}\NormalTok{,}
                                                     \StringTok{"50%"}\NormalTok{,}\StringTok{"75%"}\NormalTok{,}\StringTok{"95%"}\NormalTok{,}\StringTok{"100%"}\NormalTok{))}
\end{Highlighting}
\end{Shaded}

Let's explain whats going on here. In order for GGplot to get the
grouped bar plot to work, it needs to know effectively two x elements of
each y. Here, the year part of this data frame is the group that the
each observation is a part of, and the perwatched is identifying our
percentage watched. The two dataframes are essentially the same thing,
just the second one is very long and not easily readable. However, it
enables us to plot a grouped bar plot that allows us to easily visualise
the differences in year.

\begin{Shaded}
\begin{Highlighting}[]
\KeywordTok{ggplot}\NormalTok{(datas, }\KeywordTok{aes}\NormalTok{(}\DataTypeTok{fill=}\NormalTok{perwatched, }\DataTypeTok{y=}\NormalTok{values, }\DataTypeTok{x=}\NormalTok{year)) }\OperatorTok{+}\StringTok{ }
\StringTok{  }\KeywordTok{geom_bar}\NormalTok{(}\DataTypeTok{position=}\StringTok{"dodge"}\NormalTok{, }\DataTypeTok{stat=}\StringTok{"identity"}\NormalTok{) }\OperatorTok{+}\StringTok{ }\KeywordTok{labs}\NormalTok{(}\DataTypeTok{x=}\StringTok{"Year"}\NormalTok{,}\DataTypeTok{y=}\StringTok{"% of views"}\NormalTok{) }\OperatorTok{+}\StringTok{ }
\StringTok{  }\KeywordTok{labs}\NormalTok{(}\DataTypeTok{fill =} \StringTok{"% watched"}\NormalTok{)}
\end{Highlighting}
\end{Shaded}

\includegraphics{report1_files/figure-latex/unnamed-chunk-6-1.pdf} This
is a very useful plot for our comparison. We can clearly see that
although there are slight differences in the percentages across the
years, the trend is a very similar one. There is generally a consistent
drop in viewers as the video goes on. The most significant drops
certainly are from 95\% watched to 100\% watched, showing that a decent
group of students will watch most of the video, but tune out before
watching it all. This could possibly indicate that the videos could be
slightly shortened. Youtube analytics suggest that in a given video it
is normal to see a gradual decline in audience engagement as a video
goes on, but sharper decreases like at the end of this video suggest
that the something in the video is causing many viewers to stop
watching. Source:
\url{https://creatoracademy.youtube.com/page/lesson/engagement-analytics?cid=analytics-series\&hl=en\#strategies-zippy-link-1}

\hypertarget{overall-views}{%
\subsection{Overall views}\label{overall-views}}

The analysis that looks simply at the overall percentages for a year is
all well and good, but unfortunately in this course not all videos or
years were created equal. There are big differences between years in
terms of number of students enrolled, and differences between videos in
terms of the number of views on each video. Comparisons between videos
will be tackled in the next part of this report, but for now lets stay
on video engagement across the years.

The percentage analysis we've done thus far was good for investigating
if there was a trend across years as it put all the years on a level
playing field for comparison. But, to repeat myself, \emph{not all years
are created equal}. To see the overall picture in terms of total views,
I needed to alter the calculation I was making for the percentage views.
The benefit of my approach for the first part is that we can essentially
reproduce the analysis we just did for the percentages, but simply alter
our calculations to work out views rather than percentages of views.

This was done by making another helper function, almost identical to the
avgper one used previously.

\begin{Shaded}
\begin{Highlighting}[]
\NormalTok{helper.avgview <-}\StringTok{ }\ControlFlowTok{function}\NormalTok{(y)}
\NormalTok{\{}
\NormalTok{  temp =}\StringTok{ }\KeywordTok{rep}\NormalTok{(}\OtherTok{NULL}\NormalTok{,}\DecValTok{7}\NormalTok{)}
  \ControlFlowTok{for}\NormalTok{ (i }\ControlFlowTok{in} \DecValTok{9}\OperatorTok{:}\DecValTok{15}\NormalTok{)}
\NormalTok{  \{}
\NormalTok{    temp[i}\DecValTok{-8}\NormalTok{] =}\StringTok{ }\KeywordTok{sum}\NormalTok{((y[i]}\OperatorTok{/}\DecValTok{100}\NormalTok{)}\OperatorTok{*}\NormalTok{y[}\DecValTok{4}\NormalTok{])}\OperatorTok{/}\DecValTok{13}
\NormalTok{  \}}
  \KeywordTok{return}\NormalTok{(temp)}
\NormalTok{\}}
\end{Highlighting}
\end{Shaded}

Avgview simply does the same calculation but using the total views we
can work out how many views crossed each threshold per year. We now
simply use this function to get our vectors of views.

\begin{Shaded}
\begin{Highlighting}[]
\NormalTok{y3a=}\KeywordTok{helper.avgview}\NormalTok{(y3)}
\NormalTok{y4a=}\KeywordTok{helper.avgview}\NormalTok{(y4)}
\NormalTok{y5a=}\KeywordTok{helper.avgview}\NormalTok{(y5)}
\NormalTok{y6a=}\KeywordTok{helper.avgview}\NormalTok{(y6)}
\NormalTok{y7a=}\KeywordTok{helper.avgview}\NormalTok{(y7)}
\end{Highlighting}
\end{Shaded}

We can form the data frame again for readability. This uses the same
code as before when forming the data frame so most of it has been
committed for brevity.

\begin{Shaded}
\begin{Highlighting}[]
\NormalTok{(}\DataTypeTok{viddf =} \KeywordTok{data.frame}\NormalTok{(}\DataTypeTok{year=}\DecValTok{3}\OperatorTok{:}\DecValTok{7}\NormalTok{,}\StringTok{"5%"}\NormalTok{=a5p,}\StringTok{"10%"}\NormalTok{=a10p,}\StringTok{"25%"}\NormalTok{=a25p,}
                    \StringTok{"50%"}\NormalTok{=a50p,}\StringTok{"75%"}\NormalTok{=a75p,}\StringTok{"95%"}\NormalTok{=a95p,}\StringTok{"100%"}\NormalTok{=a100p))}
\end{Highlighting}
\end{Shaded}

\begin{verbatim}
##   year      X5.     X10.     X25.     X50.     X75.     X95.    X100.
## 1    3 552.6984 541.9287 525.4658 504.5362 489.8368 472.4660 416.1434
## 2    4 589.6921 576.6123 555.3760 529.4631 509.9937 493.5414 440.9319
## 3    5 637.6308 624.5319 601.8566 579.3067 563.3838 548.3748 484.7548
## 4    6 355.6128 345.4624 330.3835 312.6920 301.3063 291.4648 252.8473
## 5    7 306.5378 298.5349 282.4682 267.9933 258.6968 250.3831 227.6115
\end{verbatim}

We see a similar decreasing pattern as with the percentages, but here we
see the actual effect on the views.

Now in order to plot we form another data frame for the grouped bar
plot. Again, this is done with exactly the same code as before so is
omitted.

Once this is done, we plot our data.

\begin{Shaded}
\begin{Highlighting}[]
\KeywordTok{ggplot}\NormalTok{(datas, }\KeywordTok{aes}\NormalTok{(}\DataTypeTok{fill=}\NormalTok{perwatched, }\DataTypeTok{y=}\NormalTok{values, }\DataTypeTok{x=}\NormalTok{year)) }\OperatorTok{+}\StringTok{ }
\StringTok{  }\KeywordTok{geom_bar}\NormalTok{(}\DataTypeTok{position=}\StringTok{"dodge"}\NormalTok{, }\DataTypeTok{stat=}\StringTok{"identity"}\NormalTok{) }\OperatorTok{+}\StringTok{ }\KeywordTok{labs}\NormalTok{(}\DataTypeTok{x=}\StringTok{"Year"}\NormalTok{,}\DataTypeTok{y=}\StringTok{"Views"}\NormalTok{) }\OperatorTok{+}\StringTok{ }
\StringTok{  }\KeywordTok{labs}\NormalTok{(}\DataTypeTok{fill =} \StringTok{"% watched"}\NormalTok{)}
\end{Highlighting}
\end{Shaded}

\includegraphics{report1_files/figure-latex/unnamed-chunk-12-1.pdf}

Here we can see a more overall picture of what is happening across the
years. It is reflective of the situation of enrollments, which fall from
3544 in year 5 to 2342 in year 7. When you take this into account, the
apparent decrease in overall views is probably expected. We see the same
pattern as we saw with the overall percentages, the decreasing trend
across years as students continue to lose interest the further they get
into the video. As a tool to compare across the years this plot is less
useful than the percentage of views plot, as the years have a different
number of students enrolled. This will obviously have an effect on the
number of views the videos get.

\hypertarget{video-comparison}{%
\section{Video comparison}\label{video-comparison}}

Comparing between the videos presents an interesting discussion about
the use of the two plots I have looked at so far. When comparing across
years, we are less interested in total views and more looking for
trends, and this meant that the percentage graph was of more use to us
to fairly compare the years. This is because here I am mainly interested
in looking at the video stats, rather than enrollment numbers which
define the total view graph. However I think its fair to say the
opposite is true when we are looking to compare across videos. The
percentage graph will be helpful to see if there is any particular trend
with an individual video, but the total views graph should help to paint
a picture of how much a video is watched.

First, lets take a step back and simply look at across all the years how
many views each video got. This is a simpler undertaking, we just take
the values and sum them from each year for each video, then set up our
data frame and plot. With a simple bar plot, we don't need to bother
with the different levels that we needed for the grouped bar plot.
Essentially, this code is more concise and easier to understand.

\begin{Shaded}
\begin{Highlighting}[]
\NormalTok{views =}\StringTok{ }\KeywordTok{rep}\NormalTok{(}\OtherTok{NULL}\NormalTok{,}\DecValTok{13}\NormalTok{)}
\ControlFlowTok{for}\NormalTok{ (j }\ControlFlowTok{in} \DecValTok{1}\OperatorTok{:}\DecValTok{13}\NormalTok{)}
\NormalTok{\{}
  \CommentTok{#for video with row index j, get the total views (column 4) and add them up}
\NormalTok{  views[j] =}\StringTok{ }\NormalTok{y3[j,}\DecValTok{4}\NormalTok{] }\OperatorTok{+}\StringTok{ }\NormalTok{y4[j,}\DecValTok{4}\NormalTok{] }\OperatorTok{+}\StringTok{ }\NormalTok{y5[j,}\DecValTok{4}\NormalTok{] }\OperatorTok{+}\StringTok{ }\NormalTok{y6[j,}\DecValTok{4}\NormalTok{] }\OperatorTok{+}\StringTok{ }\NormalTok{y7[j,}\DecValTok{4}\NormalTok{]}
\NormalTok{\}}
\NormalTok{v =}\StringTok{ }\KeywordTok{c}\NormalTok{(}\StringTok{"1.1"}\NormalTok{,}\StringTok{"1.14"}\NormalTok{,}\StringTok{"1.17"}\NormalTok{,}\StringTok{"1.19"}\NormalTok{,}\StringTok{"1.5"}\NormalTok{,}\StringTok{"2.1"}\NormalTok{,}\StringTok{"2.11"}\NormalTok{,}
      \StringTok{"2.17"}\NormalTok{,}\StringTok{"2.4"}\NormalTok{,}\StringTok{"3.1"}\NormalTok{,}\StringTok{"3.14"}\NormalTok{,}\StringTok{"3.15"}\NormalTok{,}\StringTok{"3.2"}\NormalTok{)}
\NormalTok{viewdf =}\StringTok{ }\KeywordTok{data.frame}\NormalTok{(}\DataTypeTok{video=}\NormalTok{v, }\DataTypeTok{views=}\NormalTok{views)}
\KeywordTok{ggplot}\NormalTok{(}\DataTypeTok{data=}\NormalTok{viewdf, }\KeywordTok{aes}\NormalTok{(}\DataTypeTok{x=}\NormalTok{video, }\DataTypeTok{y=}\NormalTok{views)) }\OperatorTok{+}\StringTok{ }
\StringTok{  }\KeywordTok{geom_bar}\NormalTok{(}\DataTypeTok{stat=}\StringTok{"identity"}\NormalTok{, }\DataTypeTok{fill=}\StringTok{"steelblue"}\NormalTok{) }\OperatorTok{+}\StringTok{ }
\StringTok{  }\KeywordTok{labs}\NormalTok{(}\DataTypeTok{x=}\StringTok{"Video"}\NormalTok{,}\DataTypeTok{y=}\StringTok{"Views"}\NormalTok{)}
\end{Highlighting}
\end{Shaded}

\includegraphics{report1_files/figure-latex/unnamed-chunk-13-1.pdf}

Here we can see that as would be expected, the first video ``Welcome''
is far and away the most viewed video. There is a sharp decrease to the
next video, with the only other video that gets close in terms of views
being video 1.5 ``Privacy online and offline''. There is a clear tail
off towards the end of the course, but it seems that it levels out
towards the very end. This is a trend that is often evidenced in many
modules of undergraduate maths lecture attendance to my own experience.

Lets break this down with our percentage watched thresholds. This time
we'll first consider the total views first.

\begin{Shaded}
\begin{Highlighting}[]
\NormalTok{temp =}\StringTok{ }\KeywordTok{rep}\NormalTok{(}\OtherTok{NULL}\NormalTok{,}\DecValTok{7}\NormalTok{)}
\NormalTok{val =}\StringTok{ }\KeywordTok{c}\NormalTok{()}
\CommentTok{#for each video i}
\ControlFlowTok{for}\NormalTok{ (i }\ControlFlowTok{in} \DecValTok{1}\OperatorTok{:}\DecValTok{13}\NormalTok{)}
\NormalTok{\{}
  \CommentTok{#for each percentage watched threshold p}
  \ControlFlowTok{for}\NormalTok{ (p }\ControlFlowTok{in} \DecValTok{1}\OperatorTok{:}\DecValTok{7}\NormalTok{)}
\NormalTok{  \{}
    \CommentTok{#total views are calculated by ((percentaged watched to this point)/100)*total views}
    \CommentTok{#needs to be /100 as the 50% is stored just as 50 in the data}
\NormalTok{    temp[p] =}\StringTok{ }\NormalTok{(y3[i,p}\OperatorTok{+}\DecValTok{8}\NormalTok{]}\OperatorTok{/}\DecValTok{100}\NormalTok{)}\OperatorTok{*}\NormalTok{y3[i,}\DecValTok{4}\NormalTok{] }\OperatorTok{+}\StringTok{ }\NormalTok{(y4[i,p}\OperatorTok{+}\DecValTok{8}\NormalTok{]}\OperatorTok{/}\DecValTok{100}\NormalTok{)}\OperatorTok{*}\NormalTok{y4[i,}\DecValTok{4}\NormalTok{] }\OperatorTok{+}\StringTok{ }
\StringTok{      }\NormalTok{(y5[i,p}\OperatorTok{+}\DecValTok{8}\NormalTok{]}\OperatorTok{/}\DecValTok{100}\NormalTok{)}\OperatorTok{*}\NormalTok{y5[i,}\DecValTok{4}\NormalTok{] }\OperatorTok{+}\StringTok{ }\NormalTok{(y6[i,p}\OperatorTok{+}\DecValTok{8}\NormalTok{]}\OperatorTok{/}\DecValTok{100}\NormalTok{)}\OperatorTok{*}\NormalTok{y6[i,}\DecValTok{4}\NormalTok{] }\OperatorTok{+}\StringTok{ }
\StringTok{      }\NormalTok{(y7[i,p}\OperatorTok{+}\DecValTok{8}\NormalTok{]}\OperatorTok{/}\DecValTok{100}\NormalTok{)}\OperatorTok{*}\NormalTok{y7[i,}\DecValTok{4}\NormalTok{]}
\NormalTok{  \}}
\NormalTok{  val =}\StringTok{ }\KeywordTok{append}\NormalTok{(val,temp)}
\NormalTok{\}}
\end{Highlighting}
\end{Shaded}

In this chunk I've iterated through each video and within that each
percentage watched threshold, calculating the number of views for each
video that passed each threshold. In the earlier part of my analysis
where I compared across years I presented this part in a readable data
frame output. However, we have a lot of data to represent here so I've
skipped straight to plotting it. Hence why the data is actually stored
in one long vector rather than separate ones.

You'll hopefully recognise at least the structure of the following code
by now. We are setting up a data frame for a grouped bar plot, building
it very long so that the plot knows about the two x elements of every y.
We group the percentage watched thresholds by video and plot.

\begin{Shaded}
\begin{Highlighting}[]
\NormalTok{videos =}\StringTok{ }\KeywordTok{c}\NormalTok{(}\KeywordTok{rep}\NormalTok{(}\StringTok{"1.1"}\NormalTok{,}\DecValTok{7}\NormalTok{),}\KeywordTok{rep}\NormalTok{(}\StringTok{"1.14"}\NormalTok{,}\DecValTok{7}\NormalTok{),}\KeywordTok{rep}\NormalTok{(}\StringTok{"1.17"}\NormalTok{,}\DecValTok{7}\NormalTok{),}\KeywordTok{rep}\NormalTok{(}\StringTok{"1.19"}\NormalTok{,}\DecValTok{7}\NormalTok{),}
           \KeywordTok{rep}\NormalTok{(}\StringTok{"1.5"}\NormalTok{,}\DecValTok{7}\NormalTok{),}\KeywordTok{rep}\NormalTok{(}\StringTok{"2.1"}\NormalTok{,}\DecValTok{7}\NormalTok{),}\KeywordTok{rep}\NormalTok{(}\StringTok{"2.11"}\NormalTok{,}\DecValTok{7}\NormalTok{),}\KeywordTok{rep}\NormalTok{(}\StringTok{"2.17"}\NormalTok{,}\DecValTok{7}\NormalTok{),}\KeywordTok{rep}\NormalTok{(}\StringTok{"2.4"}\NormalTok{,}\DecValTok{7}\NormalTok{),}
           \KeywordTok{rep}\NormalTok{(}\StringTok{"3.1"}\NormalTok{,}\DecValTok{7}\NormalTok{),}\KeywordTok{rep}\NormalTok{(}\StringTok{"3.14"}\NormalTok{,}\DecValTok{7}\NormalTok{),}\KeywordTok{rep}\NormalTok{(}\StringTok{"3.15"}\NormalTok{,}\DecValTok{7}\NormalTok{),}\KeywordTok{rep}\NormalTok{(}\StringTok{"3.2"}\NormalTok{,}\DecValTok{7}\NormalTok{))}
\NormalTok{perwatched =}\StringTok{ }\KeywordTok{rep}\NormalTok{(}\KeywordTok{c}\NormalTok{(}\StringTok{"5%"}\NormalTok{,}\StringTok{"10%"}\NormalTok{,}\StringTok{"25%"}\NormalTok{,}\StringTok{"50%"}\NormalTok{,}\StringTok{"75%"}\NormalTok{,}\StringTok{"95%"}\NormalTok{,}\StringTok{"100%"}\NormalTok{),}\DecValTok{13}\NormalTok{)}
\NormalTok{values =}\StringTok{ }\NormalTok{val}
\NormalTok{datas =}\StringTok{ }\KeywordTok{data.frame}\NormalTok{(videos,perwatched,values)}
\NormalTok{datas}\OperatorTok{$}\NormalTok{perwatched =}\StringTok{ }\KeywordTok{factor}\NormalTok{(datas}\OperatorTok{$}\NormalTok{perwatched, }\DataTypeTok{levels=}\KeywordTok{c}\NormalTok{(}\StringTok{"5%"}\NormalTok{,}\StringTok{"10%"}\NormalTok{,}\StringTok{"25%"}\NormalTok{,}
                                                     \StringTok{"50%"}\NormalTok{,}\StringTok{"75%"}\NormalTok{,}\StringTok{"95%"}\NormalTok{,}\StringTok{"100%"}\NormalTok{))}

\KeywordTok{ggplot}\NormalTok{(datas, }\KeywordTok{aes}\NormalTok{(}\DataTypeTok{fill=}\NormalTok{perwatched, }\DataTypeTok{y=}\NormalTok{values, }\DataTypeTok{x=}\NormalTok{videos)) }\OperatorTok{+}\StringTok{ }
\StringTok{  }\KeywordTok{geom_bar}\NormalTok{(}\DataTypeTok{position=}\StringTok{"dodge"}\NormalTok{, }\DataTypeTok{stat=}\StringTok{"identity"}\NormalTok{) }\OperatorTok{+}\StringTok{ }
\StringTok{  }\KeywordTok{labs}\NormalTok{(}\DataTypeTok{x=}\StringTok{"Video"}\NormalTok{,}\DataTypeTok{y=}\StringTok{"Views"}\NormalTok{) }\OperatorTok{+}\StringTok{ }
\StringTok{  }\KeywordTok{labs}\NormalTok{(}\DataTypeTok{fill =} \StringTok{"% watched"}\NormalTok{)}
\end{Highlighting}
\end{Shaded}

\includegraphics{report1_files/figure-latex/unnamed-chunk-15-1.pdf}

When we looked at the years comparison plot, it was dependent on
enrollments which varied year by year. When comparing across videos
however, we are not bound by this and we can see some interesting
things. We see a familiar pattern across the board in terms of the
general decrease in viewers watching further into the video. However
what is most interesting to me in this plot is the degree to which this
occurs in each video, with many exhibiting fairly distinct behaviour.
Take for example 3.14. The videos in the same chapter all represent our
typical gradual decline from 5\% to 100\%. 3.14 however displays an
incredibly large drop from 95\% to 100\% watched, indicating something
must have cause students to not watch until the end of that video. 1.17
has a similar behavior, whereas 2.17 and 2.1 is remarkably flat in terms
of their shape, meaning they managed to retain many more of those who
watched only 5\% of the video to the end in comparison with others. We
discussed how 1.5 is the video that is most popular when we discount the
welcome 1.1 video, and the trend it displays is very interesting too.
More than in most other videos, it loses viewers from the 5\% to 50\%
threshold.

In order to gain a closer of understanding of the behaviour of the data
around the videos across the years, we plot the percentage of views
across videos.

\begin{Shaded}
\begin{Highlighting}[]
\NormalTok{temp =}\StringTok{ }\KeywordTok{rep}\NormalTok{(}\OtherTok{NULL}\NormalTok{,}\DecValTok{7}\NormalTok{)}
\NormalTok{val =}\StringTok{ }\KeywordTok{c}\NormalTok{()}
\ControlFlowTok{for}\NormalTok{ (i }\ControlFlowTok{in} \DecValTok{1}\OperatorTok{:}\DecValTok{13}\NormalTok{)}
\NormalTok{\{}
  \ControlFlowTok{for}\NormalTok{ (p }\ControlFlowTok{in} \DecValTok{1}\OperatorTok{:}\DecValTok{7}\NormalTok{)}
\NormalTok{  \{}
    \CommentTok{#here we can just take the percentages and average them}
\NormalTok{    temp[p] =}\StringTok{ }\NormalTok{(y3[i,p}\OperatorTok{+}\DecValTok{8}\NormalTok{] }\OperatorTok{+}\StringTok{ }\NormalTok{y4[i,p}\OperatorTok{+}\DecValTok{8}\NormalTok{] }\OperatorTok{+}\StringTok{ }\NormalTok{y5[i,p}\OperatorTok{+}\DecValTok{8}\NormalTok{] }\OperatorTok{+}\StringTok{ }\NormalTok{y6[i,p}\OperatorTok{+}\DecValTok{8}\NormalTok{] }\OperatorTok{+}\StringTok{ }\NormalTok{y7[i,p}\OperatorTok{+}\DecValTok{8}\NormalTok{])}\OperatorTok{/}\DecValTok{5}
\NormalTok{  \}}
\NormalTok{  val =}\StringTok{ }\KeywordTok{append}\NormalTok{(val,temp)}
\NormalTok{\}}

\NormalTok{videos =}\StringTok{ }\KeywordTok{c}\NormalTok{(}\KeywordTok{rep}\NormalTok{(}\StringTok{"1.1"}\NormalTok{,}\DecValTok{7}\NormalTok{),}\KeywordTok{rep}\NormalTok{(}\StringTok{"1.14"}\NormalTok{,}\DecValTok{7}\NormalTok{),}\KeywordTok{rep}\NormalTok{(}\StringTok{"1.17"}\NormalTok{,}\DecValTok{7}\NormalTok{),}\KeywordTok{rep}\NormalTok{(}\StringTok{"1.19"}\NormalTok{,}\DecValTok{7}\NormalTok{),}
           \KeywordTok{rep}\NormalTok{(}\StringTok{"1.5"}\NormalTok{,}\DecValTok{7}\NormalTok{),}\KeywordTok{rep}\NormalTok{(}\StringTok{"2.1"}\NormalTok{,}\DecValTok{7}\NormalTok{),}\KeywordTok{rep}\NormalTok{(}\StringTok{"2.11"}\NormalTok{,}\DecValTok{7}\NormalTok{),}\KeywordTok{rep}\NormalTok{(}\StringTok{"2.17"}\NormalTok{,}\DecValTok{7}\NormalTok{),}\KeywordTok{rep}\NormalTok{(}\StringTok{"2.4"}\NormalTok{,}\DecValTok{7}\NormalTok{),}
           \KeywordTok{rep}\NormalTok{(}\StringTok{"3.1"}\NormalTok{,}\DecValTok{7}\NormalTok{),}\KeywordTok{rep}\NormalTok{(}\StringTok{"3.14"}\NormalTok{,}\DecValTok{7}\NormalTok{),}\KeywordTok{rep}\NormalTok{(}\StringTok{"3.15"}\NormalTok{,}\DecValTok{7}\NormalTok{),}\KeywordTok{rep}\NormalTok{(}\StringTok{"3.2"}\NormalTok{,}\DecValTok{7}\NormalTok{))}
\NormalTok{perwatched =}\StringTok{ }\KeywordTok{rep}\NormalTok{(}\KeywordTok{c}\NormalTok{(}\StringTok{"5%"}\NormalTok{,}\StringTok{"10%"}\NormalTok{,}\StringTok{"25%"}\NormalTok{,}\StringTok{"50%"}\NormalTok{,}\StringTok{"75%"}\NormalTok{,}\StringTok{"95%"}\NormalTok{,}\StringTok{"100%"}\NormalTok{),}\DecValTok{13}\NormalTok{)}
\NormalTok{values =}\StringTok{ }\NormalTok{val}
\NormalTok{datas =}\StringTok{ }\KeywordTok{data.frame}\NormalTok{(videos,perwatched,values)}
\NormalTok{datas}\OperatorTok{$}\NormalTok{perwatched =}\StringTok{ }\KeywordTok{factor}\NormalTok{(datas}\OperatorTok{$}\NormalTok{perwatched, }\DataTypeTok{levels=}\KeywordTok{c}\NormalTok{(}\StringTok{"5%"}\NormalTok{,}\StringTok{"10%"}\NormalTok{,}\StringTok{"25%"}\NormalTok{,}
                                                     \StringTok{"50%"}\NormalTok{,}\StringTok{"75%"}\NormalTok{,}\StringTok{"95%"}\NormalTok{,}\StringTok{"100%"}\NormalTok{))}

\KeywordTok{ggplot}\NormalTok{(datas, }\KeywordTok{aes}\NormalTok{(}\DataTypeTok{fill=}\NormalTok{perwatched, }\DataTypeTok{y=}\NormalTok{values, }\DataTypeTok{x=}\NormalTok{videos)) }\OperatorTok{+}\StringTok{ }
\StringTok{  }\KeywordTok{geom_bar}\NormalTok{(}\DataTypeTok{position=}\StringTok{"dodge"}\NormalTok{, }\DataTypeTok{stat=}\StringTok{"identity"}\NormalTok{) }\OperatorTok{+}\StringTok{ }
\StringTok{  }\KeywordTok{labs}\NormalTok{(}\DataTypeTok{x=}\StringTok{"Video"}\NormalTok{,}\DataTypeTok{y=}\StringTok{"% of views"}\NormalTok{) }\OperatorTok{+}\StringTok{ }
\StringTok{  }\KeywordTok{labs}\NormalTok{(}\DataTypeTok{fill =} \StringTok{"% watched"}\NormalTok{)}
\end{Highlighting}
\end{Shaded}

\includegraphics{report1_files/figure-latex/unnamed-chunk-16-1.pdf}

This accentuates the patterns we saw in the previous plot. 3.14 loses
almost 20\% of its audience that watched 95\% of the video before the
end. Considering most videos aren't losing that proportion of its
audience across the whole video, this further reinforces the implication
that something in that video is making users not watch the last 5\% of
the video.

\hypertarget{effect-of-duration}{%
\subsection{Effect of duration}\label{effect-of-duration}}

Its a reasonable question to ask if duration is having an effect on the
``watchability'' of a video, as intuitively, longer videos may require
more engagement. To investigate this, we'll use a scatter plot.

\begin{Shaded}
\begin{Highlighting}[]
\NormalTok{duration =}\StringTok{ }\KeywordTok{rep}\NormalTok{(}\OtherTok{NULL}\NormalTok{,}\DecValTok{13}\NormalTok{)}
\ControlFlowTok{for}\NormalTok{ (q }\ControlFlowTok{in} \DecValTok{1}\OperatorTok{:}\DecValTok{13}\NormalTok{)}
\NormalTok{\{}
  \CommentTok{#duration of videos is the same across years3-7 so we only need to grab from one year}
\NormalTok{  duration[q] =}\StringTok{ }\NormalTok{y3[q,}\DecValTok{3}\NormalTok{]}
\NormalTok{\}}
\NormalTok{viewdf}\OperatorTok{$}\NormalTok{duration =}\StringTok{ }\NormalTok{duration}

\KeywordTok{ggplot}\NormalTok{(}\DataTypeTok{data=}\NormalTok{viewdf, }\KeywordTok{aes}\NormalTok{(}\DataTypeTok{x=}\NormalTok{duration,}\DataTypeTok{y=}\NormalTok{views)) }\OperatorTok{+}\StringTok{ }\KeywordTok{geom_point}\NormalTok{(}\DataTypeTok{size=}\DecValTok{3}\NormalTok{,}\DataTypeTok{colour=}\StringTok{"steelblue"}\NormalTok{) }\OperatorTok{+}\StringTok{ }
\StringTok{  }\KeywordTok{geom_text}\NormalTok{(}\DataTypeTok{label=}\NormalTok{viewdf}\OperatorTok{$}\NormalTok{video,}\DataTypeTok{nudge_y =} \DecValTok{200}\NormalTok{)}
\end{Highlighting}
\end{Shaded}

\includegraphics{report1_files/figure-latex/unnamed-chunk-17-1.pdf}

This has been done as an addition to the data frame we used for the
simple bar plot of videos against views. Here we can see that there's no
real overall pattern to the plot, so we cant really say anything about
duration having an effect on viewership of videos.

When we observe this in conjunction with the analysis we've conducted
before on the videos, we can pull out some interesting ideas. The
cluster in the bottom left of the plot represents the shortest videos
that are getting a slightly less than average amount of viewers. When we
go back and look at the pattern of their viewers engagement, we see that
these are the videos with probably the least viewers lost throughout the
video. This is insufficient evidence to make any sort of conclusion
about shorter videos keeping their audience, as we haven't accounted for
the possibility that less viewed videos have a more committed audience.
We can see that the longest video, 2.4, has about average views and a
normal pattern to viewership engagement. In fact most of the longer
videos show the more linear decline in viewers as the videos go on.

\hypertarget{region-analysis}{%
\section{Region analysis}\label{region-analysis}}

Here I'll investigate the habits of users from different continents when
watching videos. First, we'll just have a look at which continents make
up for the different shares of views. These figures are pulled from each
of the year files and are calculated in a similar fashion as before.

\begin{Shaded}
\begin{Highlighting}[]
\CommentTok{#the continent %s are in columns 22 to 27}
\CommentTok{#performing the same calculations for total views as before}
\CommentTok{#just this time we're using the continent %s rather than viewed %s}
\NormalTok{europe =}\StringTok{ }\NormalTok{(y3[[}\DecValTok{22}\NormalTok{]]}\OperatorTok{/}\DecValTok{100}\NormalTok{)}\OperatorTok{*}\NormalTok{y3[[}\DecValTok{4}\NormalTok{]] }\OperatorTok{+}\StringTok{ }\NormalTok{(y4[[}\DecValTok{22}\NormalTok{]]}\OperatorTok{/}\DecValTok{100}\NormalTok{)}\OperatorTok{*}\NormalTok{y4[[}\DecValTok{4}\NormalTok{]] }\OperatorTok{+}\StringTok{ }\NormalTok{(y5[[}\DecValTok{22}\NormalTok{]]}\OperatorTok{/}\DecValTok{100}\NormalTok{)}\OperatorTok{*}\NormalTok{y5[[}\DecValTok{4}\NormalTok{]] }\OperatorTok{+}\StringTok{ }
\StringTok{  }\NormalTok{(y6[[}\DecValTok{22}\NormalTok{]]}\OperatorTok{/}\DecValTok{100}\NormalTok{)}\OperatorTok{*}\NormalTok{y6[[}\DecValTok{4}\NormalTok{]] }\OperatorTok{+}\StringTok{ }\NormalTok{(y7[[}\DecValTok{22}\NormalTok{]]}\OperatorTok{/}\DecValTok{100}\NormalTok{)}\OperatorTok{*}\NormalTok{y7[[}\DecValTok{4}\NormalTok{]]}
\NormalTok{oce =}\StringTok{ }\NormalTok{(y3[[}\DecValTok{23}\NormalTok{]]}\OperatorTok{/}\DecValTok{100}\NormalTok{)}\OperatorTok{*}\NormalTok{y3[[}\DecValTok{4}\NormalTok{]] }\OperatorTok{+}\StringTok{ }\NormalTok{(y4[[}\DecValTok{23}\NormalTok{]]}\OperatorTok{/}\DecValTok{100}\NormalTok{)}\OperatorTok{*}\NormalTok{y4[[}\DecValTok{4}\NormalTok{]] }\OperatorTok{+}\StringTok{ }\NormalTok{(y5[[}\DecValTok{23}\NormalTok{]]}\OperatorTok{/}\DecValTok{100}\NormalTok{)}\OperatorTok{*}\NormalTok{y5[[}\DecValTok{4}\NormalTok{]] }\OperatorTok{+}\StringTok{ }
\StringTok{  }\NormalTok{(y6[[}\DecValTok{23}\NormalTok{]]}\OperatorTok{/}\DecValTok{100}\NormalTok{)}\OperatorTok{*}\NormalTok{y6[[}\DecValTok{4}\NormalTok{]] }\OperatorTok{+}\StringTok{ }\NormalTok{(y7[[}\DecValTok{23}\NormalTok{]]}\OperatorTok{/}\DecValTok{100}\NormalTok{)}\OperatorTok{*}\NormalTok{y7[[}\DecValTok{4}\NormalTok{]]}
\NormalTok{asia =}\StringTok{ }\NormalTok{(y3[[}\DecValTok{24}\NormalTok{]]}\OperatorTok{/}\DecValTok{100}\NormalTok{)}\OperatorTok{*}\NormalTok{y3[[}\DecValTok{4}\NormalTok{]] }\OperatorTok{+}\StringTok{ }\NormalTok{(y4[[}\DecValTok{24}\NormalTok{]]}\OperatorTok{/}\DecValTok{100}\NormalTok{)}\OperatorTok{*}\NormalTok{y4[[}\DecValTok{4}\NormalTok{]] }\OperatorTok{+}\StringTok{ }\NormalTok{(y5[[}\DecValTok{24}\NormalTok{]]}\OperatorTok{/}\DecValTok{100}\NormalTok{)}\OperatorTok{*}\NormalTok{y5[[}\DecValTok{4}\NormalTok{]] }\OperatorTok{+}\StringTok{ }
\StringTok{  }\NormalTok{(y6[[}\DecValTok{24}\NormalTok{]]}\OperatorTok{/}\DecValTok{100}\NormalTok{)}\OperatorTok{*}\NormalTok{y6[[}\DecValTok{4}\NormalTok{]] }\OperatorTok{+}\StringTok{ }\NormalTok{(y7[[}\DecValTok{24}\NormalTok{]]}\OperatorTok{/}\DecValTok{100}\NormalTok{)}\OperatorTok{*}\NormalTok{y7[[}\DecValTok{4}\NormalTok{]]}
\NormalTok{NorthAm =}\StringTok{ }\NormalTok{(y3[[}\DecValTok{25}\NormalTok{]]}\OperatorTok{/}\DecValTok{100}\NormalTok{)}\OperatorTok{*}\NormalTok{y3[[}\DecValTok{4}\NormalTok{]] }\OperatorTok{+}\StringTok{ }\NormalTok{(y4[[}\DecValTok{25}\NormalTok{]]}\OperatorTok{/}\DecValTok{100}\NormalTok{)}\OperatorTok{*}\NormalTok{y4[[}\DecValTok{4}\NormalTok{]] }\OperatorTok{+}\StringTok{ }\NormalTok{(y5[[}\DecValTok{25}\NormalTok{]]}\OperatorTok{/}\DecValTok{100}\NormalTok{)}\OperatorTok{*}\NormalTok{y5[[}\DecValTok{4}\NormalTok{]] }\OperatorTok{+}\StringTok{ }
\StringTok{  }\NormalTok{(y6[[}\DecValTok{25}\NormalTok{]]}\OperatorTok{/}\DecValTok{100}\NormalTok{)}\OperatorTok{*}\NormalTok{y6[[}\DecValTok{4}\NormalTok{]] }\OperatorTok{+}\StringTok{ }\NormalTok{(y7[[}\DecValTok{25}\NormalTok{]]}\OperatorTok{/}\DecValTok{100}\NormalTok{)}\OperatorTok{*}\NormalTok{y7[[}\DecValTok{4}\NormalTok{]]}
\NormalTok{SouthAm =}\StringTok{ }\NormalTok{(y3[[}\DecValTok{26}\NormalTok{]]}\OperatorTok{/}\DecValTok{100}\NormalTok{)}\OperatorTok{*}\NormalTok{y3[[}\DecValTok{4}\NormalTok{]] }\OperatorTok{+}\StringTok{ }\NormalTok{(y4[[}\DecValTok{26}\NormalTok{]]}\OperatorTok{/}\DecValTok{100}\NormalTok{)}\OperatorTok{*}\NormalTok{y4[[}\DecValTok{4}\NormalTok{]] }\OperatorTok{+}\StringTok{ }\NormalTok{(y5[[}\DecValTok{26}\NormalTok{]]}\OperatorTok{/}\DecValTok{100}\NormalTok{)}\OperatorTok{*}\NormalTok{y5[[}\DecValTok{4}\NormalTok{]] }\OperatorTok{+}\StringTok{ }
\StringTok{  }\NormalTok{(y6[[}\DecValTok{26}\NormalTok{]]}\OperatorTok{/}\DecValTok{100}\NormalTok{)}\OperatorTok{*}\NormalTok{y6[[}\DecValTok{4}\NormalTok{]] }\OperatorTok{+}\StringTok{ }\NormalTok{(y7[[}\DecValTok{26}\NormalTok{]]}\OperatorTok{/}\DecValTok{100}\NormalTok{)}\OperatorTok{*}\NormalTok{y7[[}\DecValTok{4}\NormalTok{]]}
\NormalTok{africa =}\StringTok{ }\NormalTok{(y3[[}\DecValTok{27}\NormalTok{]]}\OperatorTok{/}\DecValTok{100}\NormalTok{)}\OperatorTok{*}\NormalTok{y3[[}\DecValTok{4}\NormalTok{]] }\OperatorTok{+}\StringTok{ }\NormalTok{(y4[[}\DecValTok{27}\NormalTok{]]}\OperatorTok{/}\DecValTok{100}\NormalTok{)}\OperatorTok{*}\NormalTok{y4[[}\DecValTok{4}\NormalTok{]] }\OperatorTok{+}\StringTok{ }\NormalTok{(y5[[}\DecValTok{27}\NormalTok{]]}\OperatorTok{/}\DecValTok{100}\NormalTok{)}\OperatorTok{*}\NormalTok{y5[[}\DecValTok{4}\NormalTok{]] }\OperatorTok{+}\StringTok{ }
\StringTok{  }\NormalTok{(y6[[}\DecValTok{27}\NormalTok{]]}\OperatorTok{/}\DecValTok{100}\NormalTok{)}\OperatorTok{*}\NormalTok{y6[[}\DecValTok{4}\NormalTok{]] }\OperatorTok{+}\StringTok{ }\NormalTok{(y7[[}\DecValTok{27}\NormalTok{]]}\OperatorTok{/}\DecValTok{100}\NormalTok{)}\OperatorTok{*}\NormalTok{y7[[}\DecValTok{4}\NormalTok{]]}
\end{Highlighting}
\end{Shaded}

These vectors calculate the share each region has of the total views of
each video. This will be useful for the next plot we do after this. For
now, we need to sum these to get our region bar plot.

\begin{Shaded}
\begin{Highlighting}[]
\NormalTok{values =}\StringTok{ }\KeywordTok{c}\NormalTok{(}\KeywordTok{sum}\NormalTok{(europe),}\KeywordTok{sum}\NormalTok{(oce),}\KeywordTok{sum}\NormalTok{(asia),}\KeywordTok{sum}\NormalTok{(NorthAm),}\KeywordTok{sum}\NormalTok{(SouthAm),}\KeywordTok{sum}\NormalTok{(africa))}
\NormalTok{continents =}\StringTok{ }\KeywordTok{c}\NormalTok{(}\StringTok{"Europe"}\NormalTok{, }\StringTok{"Oceania"}\NormalTok{, }\StringTok{"Asia"}\NormalTok{, }
               \StringTok{"North America"}\NormalTok{, }\StringTok{"South America"}\NormalTok{, }\StringTok{"Africa"}\NormalTok{)}
\NormalTok{cont =}\StringTok{ }\KeywordTok{data.frame}\NormalTok{(}\DataTypeTok{continent =}\NormalTok{ continents, }\DataTypeTok{views =}\NormalTok{ values)}
\KeywordTok{ggplot}\NormalTok{(}\DataTypeTok{data=}\NormalTok{cont, }\KeywordTok{aes}\NormalTok{(}\DataTypeTok{x=}\NormalTok{continent, }\DataTypeTok{y=}\NormalTok{views)) }\OperatorTok{+}\StringTok{ }
\StringTok{  }\KeywordTok{geom_bar}\NormalTok{(}\DataTypeTok{stat=}\StringTok{"identity"}\NormalTok{, }\DataTypeTok{fill=}\StringTok{"steelblue"}\NormalTok{) }\OperatorTok{+}\StringTok{ }
\StringTok{  }\KeywordTok{labs}\NormalTok{(}\DataTypeTok{x=}\StringTok{"Continent"}\NormalTok{,}\DataTypeTok{y=}\StringTok{"Views"}\NormalTok{)}
\end{Highlighting}
\end{Shaded}

\includegraphics{report1_files/figure-latex/unnamed-chunk-19-1.pdf}

Europe clearly dominates the overall views since the course has been
collecting video stats, as you would expect from briefly investigating
the enrollment numbers. The next most popular continents of Asia, North
America and Africa are also not surprising.

To further investigate, let's look into if there are differences between
videos in terms of where the views are coming from. Here I am going to
simplify our continents into ``European'' and ``Non-European'', just to
simplify our comparisons. It turns out that these groups roughly
represent about a 60:40 split of the total views (Europe represents 58\%
= 23680 views, Non European represents 42\% = 17165). Therefore we total
up all of the rest of the continents into one vector and create a
grouped bar plot to represent how these two differing groups engage with
each video.

\begin{Shaded}
\begin{Highlighting}[]
\NormalTok{noneurope =}\StringTok{ }\NormalTok{oce}\OperatorTok{+}\NormalTok{asia}\OperatorTok{+}\NormalTok{NorthAm}\OperatorTok{+}\NormalTok{SouthAm}\OperatorTok{+}\NormalTok{africa}
\NormalTok{values =}\StringTok{ }\KeywordTok{c}\NormalTok{(europe, noneurope)}
\NormalTok{video =}\StringTok{ }\KeywordTok{rep}\NormalTok{(}\KeywordTok{c}\NormalTok{(}\StringTok{"1.1"}\NormalTok{,}\StringTok{"1.14"}\NormalTok{,}\StringTok{"1.17"}\NormalTok{,}\StringTok{"1.19"}\NormalTok{,}\StringTok{"1.5"}\NormalTok{,}\StringTok{"2.1"}\NormalTok{,}\StringTok{"2.11"}\NormalTok{,}
              \StringTok{"2.17"}\NormalTok{,}\StringTok{"2.4"}\NormalTok{,}\StringTok{"3.1"}\NormalTok{,}\StringTok{"3.14"}\NormalTok{,}\StringTok{"3.15"}\NormalTok{,}\StringTok{"3.2"}\NormalTok{),}\DecValTok{2}\NormalTok{)}
\NormalTok{european =}\StringTok{ }\KeywordTok{c}\NormalTok{(}\KeywordTok{rep}\NormalTok{(}\StringTok{"Europe"}\NormalTok{,}\DecValTok{13}\NormalTok{),}\KeywordTok{rep}\NormalTok{(}\StringTok{"Non-European"}\NormalTok{,}\DecValTok{13}\NormalTok{))}
\NormalTok{dat =}\StringTok{ }\KeywordTok{data.frame}\NormalTok{(video,european,values)}
\KeywordTok{ggplot}\NormalTok{(dat, }\KeywordTok{aes}\NormalTok{(}\DataTypeTok{fill=}\NormalTok{video, }\DataTypeTok{y=}\NormalTok{values, }\DataTypeTok{x=}\NormalTok{european)) }\OperatorTok{+}\StringTok{ }
\StringTok{  }\KeywordTok{geom_bar}\NormalTok{(}\DataTypeTok{position=}\StringTok{"dodge"}\NormalTok{, }\DataTypeTok{stat=}\StringTok{"identity"}\NormalTok{) }\OperatorTok{+}\StringTok{ }\KeywordTok{labs}\NormalTok{(}\DataTypeTok{x=}\StringTok{"Continent"}\NormalTok{,}\DataTypeTok{y=}\StringTok{"Views"}\NormalTok{) }\OperatorTok{+}\StringTok{ }
\StringTok{  }\KeywordTok{labs}\NormalTok{(}\DataTypeTok{fill =} \StringTok{"Video"}\NormalTok{)}
\end{Highlighting}
\end{Shaded}

\includegraphics{report1_files/figure-latex/unnamed-chunk-20-1.pdf}

This plot helps us to separate the trends of the European students and
Non-European students. We can see a fairly similar pattern between the
groups, with a few noticeable differences. Non-Europeans responsible for
a larger proportion of the initial 1.1 welcome video views, and in
general less for the other videos. European students comparatively are
viewing the welcome video less, but in general engaging with the other
videos more.

To further investigate this, we can look at this as a stacked bar plot
of the video views.

\begin{Shaded}
\begin{Highlighting}[]
\KeywordTok{ggplot}\NormalTok{(dat, }\KeywordTok{aes}\NormalTok{(}\DataTypeTok{fill=}\NormalTok{european, }\DataTypeTok{y=}\NormalTok{values, }\DataTypeTok{x=}\NormalTok{video)) }\OperatorTok{+}\StringTok{ }
\StringTok{  }\KeywordTok{geom_bar}\NormalTok{(}\DataTypeTok{position=}\StringTok{"stack"}\NormalTok{, }\DataTypeTok{stat=}\StringTok{"identity"}\NormalTok{) }\OperatorTok{+}\StringTok{ }\KeywordTok{labs}\NormalTok{(}\DataTypeTok{x=}\StringTok{"Video"}\NormalTok{,}\DataTypeTok{y=}\StringTok{"Views"}\NormalTok{) }\OperatorTok{+}\StringTok{ }
\StringTok{  }\KeywordTok{labs}\NormalTok{(}\DataTypeTok{fill =} \StringTok{"Continent"}\NormalTok{)}
\end{Highlighting}
\end{Shaded}

\includegraphics{report1_files/figure-latex/unnamed-chunk-21-1.pdf}

This plot allows us to get the picture when focusing more on individual
video views. We can also plot the stacked percentage plot to see a
comparative effect.

\begin{Shaded}
\begin{Highlighting}[]
\KeywordTok{ggplot}\NormalTok{(dat, }\KeywordTok{aes}\NormalTok{(}\DataTypeTok{fill=}\NormalTok{european, }\DataTypeTok{y=}\NormalTok{values, }\DataTypeTok{x=}\NormalTok{video)) }\OperatorTok{+}\StringTok{ }
\StringTok{  }\KeywordTok{geom_bar}\NormalTok{(}\DataTypeTok{position=}\StringTok{"fill"}\NormalTok{, }\DataTypeTok{stat=}\StringTok{"identity"}\NormalTok{) }\OperatorTok{+}\StringTok{ }\KeywordTok{labs}\NormalTok{(}\DataTypeTok{x=}\StringTok{"Video"}\NormalTok{,}\DataTypeTok{y=}\StringTok{"Views"}\NormalTok{) }\OperatorTok{+}\StringTok{ }
\StringTok{  }\KeywordTok{labs}\NormalTok{(}\DataTypeTok{fill =} \StringTok{"Continent"}\NormalTok{)}
\end{Highlighting}
\end{Shaded}

\includegraphics{report1_files/figure-latex/unnamed-chunk-22-1.pdf}

This shows us without doubt that the Europeans dominate the views of
most videos, but in a video like 1.5, Non Europeans make up 45\% of
views which is slightly more than their overall share of video views.
What we can draw from this is that Non-Europeans are much more likely to
watch the 1.1 welcome video, but from there the mix is fairly balanced
in terms of views compared against number of enrolements.

Overall we can see from the regional analysis that video views are
dominated by the European students due to them being the most
represented group in the enrolements. Certain videos are more viewed by
Non-European students, but generally videos are engaged with
proportionally across the board.

\end{document}
